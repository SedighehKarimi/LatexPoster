%%created by R. Rietbroek 11 March 2011
%%A0 Poster template in  IGG style
%% Updated Sept 2017 to new corporate design of Uni Bonn
%% This is essentially a collection of stolen artwork/ideas :) Credits to
%%Indridi Einarsson, Enrico Kurtenbach and the PGF/tikz wizard till Tantau

%\documentclass[landscape,a0b,final,a4resizeable]{a0poster}
%\documentclass[landscape,a0,final]{a0poster}
%\documentclass[portrait,a0b,final,a4resizeable]{a0poster}
\documentclass[portrait,a0,final]{a0poster}
%%% Option "a4resizeable" makes it possible ot resize the
%   poster by the command: psresize -pa4 poster.ps poster-a4.ps
%   For final printing, please remove option "a4resizeable" !!

%%%%%%%%%%%%%%%%%%%%%%%%%%%%%%%%%%%%%%%%%%%%%%
%%%%%%%%%%% required packages %%%%%%%%%%%%%%
%%%%%%%%%%%%%%%%%%%%%%%%%%%%%%%%%%%%%%%%%%%%%%%
\usepackage{ifxetex}

\usepackage{url}


\ifxetex

\usepackage[xetex,colorlinks=true,linkcolor=black,citecolor=black,urlcolor=black]{hyperref}
%\usepackage{xltxtra}


%%%%%%%%%%%%%%%%%%%%%%%%%%%%%%%%%%%%%%%%%%%%%%%%%
%%%%%%%%%%%%% Font settings %%%%%%%%%%%%%%%%%%%%
%%%%%%%%%%%%%%%%%%%%%%%%%%%%%%%%%%%%%%%%%%%%%%%%%

\usepackage[]{fontspec}
 % \usepackage{xunicode}

  \setmainfont{Carlito}
%  \setmonofont{Exo-Regular}
  %%NOTE the setting below affects \sffamily used for headers etc.
  \setsansfont{Exo 2}
\else

   \usepackage[pdftex,colorlinks=true,linkcolor=black,citecolor=black,urlcolor=black]{hyperref}
   \usepackage[utf8]{inputenc}
   \usepackage[pdftex]{graphicx}
%%%%%%%%%%%%%%%%%%%%%%%%%%%%%%%%%%%%%%%%%%%%%%%%%
%%%%%%%%%%%%% Font settings %%%%%%%%%%%%%%%%%%%%
%%%%%%%%%%%%%%%%%%%%%%%%%%%%%%%%%%%%%%%%%%%%%%%%%
   \usepackage{carlito}
\fi
                
                
%\usepackage[xetex,colorlinks=true,linkcolor=black,citecolor=black]{hyperref}
\usepackage{tikz}
\usetikzlibrary{shadows}
\usepackage{sectsty}
\usepackage[absolute,overlay]{textpos}
%\usepackage[absolute,overlay,showboxes]{textpos} % the 'showboxes' options can be handy for debugging
\usepackage{ifthen}


%%%%%%%%%%%%%%%%%%%%%%%%%%%%%%%%%%%%%%%%%%%%
%%%%%% add directory to graphics path
%%%%%%%%%%%%%%%%%%%%%%%%%%%%%%%%%%%%%%%%%
\graphicspath{{figures/}}

%%%%%%%%%%%%%%%%%%%%%%%%%%%%%%%%%%%%
%%%% Color definitions
%%%%%%%%%%%%%%%%%%%%%%%%%%%%%%%%%%%%%%%%%
\definecolor{uniblue}{rgb}{0.027451,   0.321569,   0.603922}
\definecolor{unigelb}{rgb}{0.98824,   0.72941,   0.00000}
\definecolor{unigrau}{rgb}{0.80672,   0.80672,   0.74510}
%\definecolor{gfzblue}{rgb}{0., 0.4, 0.65}
%\definecolor{unigrau}{rgb}{0.8, 0.8, 0.8}
%\definecolor{sunsetorange}{rgb}{1, .3, .11}
\definecolor{mossgreen}{rgb}{0.14,0.5,0.06}

%%%%%%%%%%%%%%%%%%%%%%%%%%%%%%%%%%%%%%%%%%%%%%%%%%%%%%%%%%%%%%%%%%%%%
%%%%%%%%%%%%% sectsty settings (section header appearance)%%%%%%%%%%
%%%%%%%%%%%%%%%%%%%%%%%%%%%%%%%%%%%%%%%%%%%%%%%%%%%%%%%%%%%%%%%%%%%%%
\allsectionsfont{\centering\sffamily\textcolor{uniblue}}


%%%%%%%%%%%%%%%%%%%%%%%%%%%%%%%%%%%%%%%%%%%%%%%%%%%%%%%%%%%%
%%%%%%%%%% Define grid for textpos %%%%%%%%%%%%%%%%%%%%%%%%%
%%%%%%%%%%%%%%%%%%%%%%%%%%%%%%%%%%%%%%%%%%%%%%%%%%%%%%%%%%%
\TPGrid{16}{16} % so page is gridded up in 16 blocks wide and 16 blocks high


%%%%%%%%%%%%%%%%%%%%%%%%%%%%%%%%%%%%%%%%%%%
%%%%% General document info/values %%%%%%%%%%%
%%%%%%%%%%%%%%%%%%%%%%%%%%%%%%%%%%%%%%%%%%%%
%set the title of the poster here:
\newcommand{\ititle}{The oh so catchy title of the poster}
%Your name
\newcommand{\iauthor}{Roelof Rietbroek}
%your email
\newcommand{\iemail}{roelof@geod.uni-bonn.de}
%your twitter account
\newcommand{\itweet}{@r\_rietje}
% your blog/homepage
\newcommand{\iblog}{http://wobbly.earth}

%set the name of your background image here
\newcommand{\backgroundimage}{posterbackground_hires_portrait} %comment this if you want a plain background

%some goodies for in the meta data of the pdf document
  \hypersetup{
    pdftitle={\ititle},
    pdfauthor={\iauthor},
    pdfsubject={Poster, Conference ... },
    pdfkeywords={Surface loading, geocenter motion,GRACE, gravimetry, GPS networks, Ocean modelling, joint inversion}
  }

%%%%%%%%%%%%%%%%%%%%%%%%%%%%%%%%%%%%%%%%%%%%%%%%%%%%%%%%%%%%%

%%%%%%%%%%%%%%%%%%%%%%%%%%%%%%%%%%%%%%%%%%%%%%%%%%%%%%%
%%%%%%%%%%%%%%%%%%%% Poster Macros %%%%%%%%%%%%%%%%
%%%%%%%%%%%%%%%%%%%%%%%%%%%%%%%%%%%%%%%%

%%%% new command to make a framed textblock at a page position (position is aligned with the center-top of the box)
\newcommand{\framedbox}[3]{
 \begin{textblock}{#1}[0.5,0.](#2)
   \noindent\begin{tikzpicture}[ultra thick,inner sep=1ex]
      \node at ( 0,0)[rectangle,draw=uniblue, line width=4pt,fill=unigrau,fill
        opacity=0.5,text opacity=1, minimum width=#1\TPHorizModule,rounded corners=2ex]{
	\begin{minipage}[c]{0.98\textwidth}
          \large#3
      \end{minipage}};
    \end{tikzpicture}
 \end{textblock}
}



%%set up the figure counter
\setcounter{figure}{0}

%%%new picture environment
\newcommand{\framedfig}[3][width=\textwidth,clip]{
  \refstepcounter{figure}
    \begin{tikzpicture}[inner sep=4.\pgflinewidth]\renewcommand{\baselinestretch}{0.6}
      \node at ( 0.,0.)[drop shadow,rectangle,draw=uniblue,fill=white,minimum width=\textwidth,rounded corners=0.ex]{
 	\begin{minipage}{0.97\textwidth}
	  \includegraphics[#1]{#2}
	  \textcolor{uniblue}{\small Fig. \arabic{figure} #3}
      \end{minipage}};
    \end{tikzpicture}
\renewcommand{\baselinestretch}{1.0}
}




%%%%%%%Customized fancy bullets%%%%%%%%%%%%%%%%%%%%
\newcommand{\fancyitem}{\item[{ \tikz\shade [ball color=uniblue]
      (0.,0.) [circular drop shadow] circle (0.3\baselineskip);}]}
\newcommand{\positem}{\item[{ \tikz\shade [ball color=mossgreen]
      (0.,0.) [circular drop shadow] circle (0.3\baselineskip);}]}
\newcommand{\negitem}{\item[{\tikz\shade [ball color=red] (0.,0.)
      [circular drop shadow] circle (0.3\baselineskip);}]}


%%%%%%%%%%%%%%%%%%%%%%%%%%%%%






%%%%%%%%%%%%%%%%%%%%%%%%%%%%%%%%%%%%%%%%%%%%%%%%%%%%%%%%%%%%%%%%%%%%%%
%%% Begin of Document
%%%%%%%%%%%%%%%%%%%%%%%%%%%%%%%%%%%%%%%%%%%%%%%%%%%%%%%%%%%%%%%%%%%%%%

\begin{document}


%%%%%%%%%%%%%%%%%%%%%%%%%%%%%%%%%%%%%%%%%%%
%%%%%%START BACKGROUND layer color/image/style or whatever you want
%%%%%%%%%%%%%%%%%%%%%%%%%%%%%%%%%%%%%%%%%%%%
%%This tikz picture layer fills the complete page and is shown behind the textpos boxes
\begin{tikzpicture}[thick,remember picture, overlay]
%define a few handy coordinates on the page
\coordinate[yshift=0.\pgflinewidth,xshift=0.\pgflinewidth] (NW) at
(current page.north west);
\coordinate[yshift=0.\pgflinewidth,xshift=0.\pgflinewidth] (SW) at
(current page.south west);
\coordinate[yshift=0.\pgflinewidth,xshift=0.\pgflinewidth] (NE) at
(current page.north east);
%IGG logo coordinate
\coordinate[yshift=-0.75cm,xshift=0.75cm] (IGGCOOR) at (NW); %IGG logo 
%uni logo coordinate
\coordinate[yshift=-0.75cm,xshift=-0.75cm] (UNICOOR) at (NE); 
% Coordinate of the top 'funny' rectangle
\coordinate[yshift=-11cm,xshift=-2cm] (RECTSTART) at (NE);

%%%%insert an image or plain color as background
\ifthenelse{\isundefined{\backgroundimage}}
{ %THEN %draw A0 outline and fill rectangle with background color
  \node  at (NW) [inner sep=0,rectangle,draw=uniblue,minimum
    width=\paperwidth-\pgflinewidth,minimum
    height=\paperheight-\pgflinewidth,anchor=north
    west,fill=unigrau]{};
}%ELSE  %insert background figure and draw A0 outline
{ 
\node  at (NW) [inner sep=0,minimum  width=\paperwidth,minimum
  height=\paperheight,anchor=north
  west]{\includegraphics[height=\paperheight,clip]{\backgroundimage}};
%draw rectangle for A0 boundary
\node  at (NW) [inner sep=0,rectangle,draw=uniblue,minimum
  width=\paperwidth-\pgflinewidth,minimum
  height=\paperheight-\pgflinewidth,anchor=north
  west]{};
}

%Insert top blue bar
\node  at (NW) [inner sep=0,rectangle,fill=uniblue,minimum
  width=\paperwidth-\pgflinewidth,minimum
  height=9cm,anchor=north west]{};

%aInsert bottom blue bar
\node  at (SW) [inner sep=0,rectangle,fill=uniblue,minimum
  width=\paperwidth-\pgflinewidth,minimum
  height=5.5cm,anchor=south west]{};

%insert IGG logo
\node at (IGGCOOR) [inner sep=0,rectangle,anchor=north
  west]{\includegraphics[height=7cm]{iggwhite}};

%insert Uni logo
\node at (UNICOOR) [inner sep=0,rectangle,anchor=north east]{\includegraphics[height=7cm]{UNI_Bonn_Logo_Invers_RZ}};

% add 7 funny rectangles with a loop
\foreach \x in {0,4,...,24}
\node at (RECTSTART) [yshift=-\x cm,inner sep=0,rectangle,anchor=base,fill=uniblue,minimum height=12mm, minimum width=12mm]{};

\end{tikzpicture}

%%%%%%%%%%END BACKGROUND layer%%%%%%%%%%%%%%



%%%%%%%%%%%From now on all text and figures need to be within textblocks!!!!!!!!!!
%%%%%%%%%%%Else they will not show (they will actually be behind the first tikz picture defined above%%

%%%%%%%%%%%%%%%%%%%%%%%%%%%%%%%%%
%%%%%%% TITLE BOX %%%%%%%%%%%%
%%%%%%%%%%%%%%%%%%%%%%%%%%%%%%%%%%
\begin{textblock}{8.5}[0.5,0.](8,0.1)
\begin{center}
{
    {\Huge \color{white} \sffamily \textbf{\ititle}\\[0.7cm]}

    {\color{unigrau} \huge \sffamily
R. Rietbroek$^{\footnotesize\textbf{1}}$,
    A. notherauthor$^{\footnotesize\textbf{2}}$}\\[0.5cm]}

    {\color{unigrau} \large \sffamily \textbf{1) Institute of Geodesy and
    Geoinformation, Bonn University 
    2) Technische Universit\"at , blah die Blah\\}}

\end{center}
\end{textblock}


%%%%%%%%%%%%%%%%%%%%%%%%%%%%%%%%%%%%%%%%
%%%%%%%%%%%% Affiliation box %%%%%%%%%
%%%%%%%%%%%%%%%%%%%%%%%%%%%%%%%%%%%%%%%%
%\begin{textblock}{10.5}[0.5,0.](8,0.9)
%\begin{center}
%{\color{unigrau} \large \sffamily \textbf{1) Institute of Geodesy and
  %Geoinformation, Bonn University 
%2) Technische Universit\"at , blah die Blah\\}}
%\end{center}
%\end{textblock}


%%%%%%%%%%%%%%%%%%%%%%%%%%%%%%%%%%%%%%%%%%%%%%
%%%%%%%% Contact details %%%%%%%%%%%%%%%%%%%%%%
%%%%%%%%%%%%%%%%%%%%%%%%%%%%%%%%%%%%%%%%%%%%

\begin{textblock}{2.8}[0.,1.](0.05,15.9)
{\large \sffamily \color{white}
  \noindent \textbf{Contact: \iauthor}\\
\includegraphics[width=8mm,clip]{Internet_Icon_white} \iblog\\
\includegraphics[width=9mm,clip]{Twitter_logo_white} \itweet\\ 
\includegraphics[width=9mm,clip]{e_mail} \iemail
}\end{textblock}

%%%%%%%%%%%%%%%%%%%%%%%%%%%%%%%%%%%%%%%%%%%%%%%%%%%%%%%%%
%%%%%%%%%%%%%%%%%%%%%% CONTENT %%%%%%%%%%%%%%%%%%%%%%%%%%
%%%%%%%%%%%%%%%%%%%%%%%%%%%%%%%%%%%%%%%%%%%%%%%%%%%%%%%%%%


%The framedbox command anchors the box on its centerline/top.
% The first argument is the width of the box (in 1/16 of the page width) and the second argument is the
% position on the page (of the anchor point). 0,0 means top left while 16,16
% means bottom right of the page. The height of the box grows
% automatically with its content.

\framedbox{4.8}{3.,1.8}{\input{IGG_examplebox}}
\framedbox{4.8}{13.,13}{\input{IGG_examplerefs}}
\framedbox{4.8}{8,10}{\input{IGG_compile}}


%%%%%%%%%%%%%%%%%%%%%%%%%%%%%%%%%%%%%%%%%%%%%%%%%%%%%%%%%%%%%%%%%%%%%%%%%%%%%%%%%%
%%%%%%%%%%%%%%%% LOWER INFO BAR ( for logos, acknowledgements etc.) %%%%%%%%%%%%%%%%
%%%%%%%%%%%%%%%%%%%%%%%%%%%%%%%%%%%%%%%%%%%%%%%%%%%%%%%%%%%%%%%%%%%%%%%%%%%%%%%%%%
\begin{textblock}{13.7}[1.,1](15.8,15.9)
    \begin{minipage}[b][4.4cm][c]{0.8\textwidth}\flushright
    \textcolor{unigrau}{{\Large  $\rightarrow$ Other Logos, Conference names etc \ldots}}
  \end{minipage}
  \begin{minipage}[b][4.4cm][c]{0.2\textwidth}\flushright
    \noindent\includegraphics[height=4.3cm,clip]{CCby}
  \end{minipage}
\end{textblock}
\end{document}
	      

